\usepackage{amsmath,amsthm,amsfonts,amssymb,amscd, amsxtra, mathrsfs}
\usepackage{url}
\usepackage{bbm}
\usepackage{color}
\usepackage{xcolor}
\usepackage{lmodern}
\usepackage{graphicx,float}
\usepackage{pdfsync}
\usepackage{lineno}
\usepackage[labelfont=bf]{caption}
\usepackage{siunitx} %% Sistema Internacional de Unidades
\usepackage{multicol}

\usepackage{changepage}
\usepackage{ragged2e}
\usepackage{biblatex}
%\usepackage{enumitem}
%\usepackage{microtype,todonotes}
\usepackage{lipsum}
\usepackage{booktabs}
\usepackage{tabularx}
\usepackage{colortbl}
\usepackage{xspace}
\usepackage{geometry}
\usepackage[scale=2]{ccicons}
%\usepackage{enumitem}
\usepackage{enumerate}
% \usepackage{subfigure}
% \usepackage{subfigmat}
\usepackage[T1]{fontenc} 
\usepackage{subfig}
% \usepackage{arev}



%\usepackage{hyperref}


%\usepackage[compact,small]{titlesec} % Muda o tamanho e a fonte do título das seções
%\setlength{\parskip}{1.2ex} % tamanho do espaço entre uma linha e um parágrafo
%\setlength{\parindent}{0em} % tamanho do espaço no inicio de um parágrafo

% As viúvas e os órfãos são linhas no início ou no final de um parágrafo que ficam penduradas 
% no topo ou no final de uma página ou coluna, separadas do resto do parágrafo.
%\clubpenalty = 10000  % orfãos
%\widowpenalty = 10000.% viuvas

% Fontes alternativas
%\usepackage{kpfonts}

%  Define as cores a serem usadas
\definecolor{UFGblue}{rgb}{0.0039, 0.3529, 0.6431} 
\definecolor{UFGred}{HTML}{990000}
\definecolor{UFGorange}{rgb}{0.9216, 0.4863, 0.0784}
\definecolor{UFGgreen}{rgb}{0.0509, 0.4509, 0.1568}
\definecolor{UFGgray}{rgb}{0.3686, 0.5255, 0.6235} % UBC Gray (secondary)
\definecolor{ultramarine}{rgb}{0, 0.125, 0.376} 
\definecolor{UFGorange}{rgb}{0.9216, 0.4863, 0.0784}%{HTML}{EB811B}
\definecolor{UFGred}{HTML}{990000}

\newcommand{\red}[1]{\textcolor{UFGred}{#1}}
\newcommand{\blue}[1]{\textcolor{UFGblue}{#1}}
\newcommand{\green}[1]{\textcolor{UFGgreen}{#1}}
\newcommand{\gray}[1]{\textcolor{UFGgray}{#1}}
\renewcommand{\leq}{\leqslant}
\renewcommand{\geq}{\geqslant}


\DeclareMathOperator{\corr}{Corr}
\DeclareMathOperator{\cov}{Cov}
\DeclareMathOperator{\var}{Var}

% definindo as margens do documento
\geometry{a4paper,text={16.5cm,25.2cm},centering}

%preambulos separados
%% \geometry{legalpaper, landscape, margin=2in}
% \geometry{a4paper, margin=1.0in}
 \geometry{a4paper,text={16.5cm,25.2cm},centering}
% \geometry{margin=2cm,nohead}
%\usepackage[left=2cm,right=2cm,top=2cm,bottom=2cm]{geometry}

%%\newtheorem{algorithm}{Algorithm}
\newtheorem{exemplo}{Exemplo}


\newcommand{\brl}{\mathrm{R}\$}
\newcommand{\R}{\mathbb{R}}

\newcommand{\sen}{\hspace{2pt}\textrm{sen \!}}
\newcommand{\senh}{\hspace{2pt}\textrm{senh \!}}


\DeclareMathOperator{\tr}{tr} % define o traço de uma matriz
\DeclareMathOperator*{\minimize}{Minimize}


% \newlist{parts}{enumerate}{3}
% \setlist[parts]{label=\arabic*.}
% \renewcommand{\part}{\item}

\renewenvironment{solution}{ {\bfseries Solução}:}{}

\newif\ifhidesolutions
\hidesolutionstrue %uncomment to hide solutions

\ifhidesolutions
\usepackage{environ}
\NewEnviron{hide}{}
\let\solution\hide
\let\endsolution\endhide
\fi



\newcommand{\red}[1]{\textcolor{UFGred}{#1}}
\newcommand{\blue}[1]{\textcolor{UFGblue}{#1}}
\newcommand{\green}[1]{\textcolor{UFGgreen}{#1}}
\newcommand{\gray}[1]{\textcolor{UFGgray}{#1}}

%\numberwithin{figure}{section}
\numberwithin{table}{section}
\newtheorem{proposition}[theorem]{Proposition}



% Colorinlistoftodos package: to insert colored comments so authors can collaborate on the content.
%\usepackage[colorinlistoftodos, textwidth=20mm, textsize=footnotesize]{todonotes}
%\newcommand{\aluno}[1]{\todo[author=\textbf{Aluno},color=green!30,caption={},inline]{#1}}
%\newcommand{\max}[1]{\todo[author=\textbf{Max},color=red!30,caption={},inline]{#1}}