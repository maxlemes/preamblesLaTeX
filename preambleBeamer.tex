\mode<presentation>
\usetheme{metropolis}
\setbeamercovered{transparent}

\newcommand{\themename}{\textbf{\textsc{metropolis}}\xspace}

\usefonttheme[onlymath]{serif}
\usefonttheme{structurebold} % fonte modo matematico


%  Define as cores a serem usadas
\definecolor{mLightBrown}{HTML}{990000} % alert block
\definecolor{mLightGreen}{HTML}{EB811B} % example text



% logo of my university
\titlegraphic{\vspace{3.5cm}\hfill\includegraphics[height=2.3cm]{figuras/ufglogo.png}}


%Colocando um marca dagua
\setbeamertemplate{background}{
  \tikz[overlay,remember picture]
  \node[opacity=.05] at (11,-5.5) {\includegraphics[width=10cm]{\string~/Projetos/ufg/figures/marcadagua.png}};
}


% - Define a cor da página
\setbeamercolor{background canvas}{bg=UFGblue!5}

% Define a cor da letra
\setbeamercolor{normal text}{fg=black}
%\setbeamercolor{example text}{fg=UFGgreen}
%\setbeamercolor{alerted text}{fg=UFGorange}


% Definindo as cores do palette
% Com cor no fundo do títulp
%\setbeamercolor{palette primary}{bg= UFGblue!30,fg=UFGblue!90!black}
%sem com no fundo do título
\setbeamercolor{palette primary}{bg= ,fg=UFGblue!90!black}
%\setbeamercolor{palette secondary}{bg=UFGblue,fg=white}
%\setbeamercolor{palette tertiary}{bg=UFGlue,fg=white}
%\setbeamercolor{palette quaternary}{bg=UFGblue,fg=white}l%


% tirando a cor dos cabeçalhos dos blocks
\setbeamercolor{block title}{bg= ,fg=UFGblue!90!black}
\setbeamercolor{block title example}{bg=}
\setbeamercolor{block title alerted}{bg=}


% Outras cores
% \setbeamercolor{structure}{fg=UFGblue} % itemize, enumerate, etc
% \setbeamercolor{section in toc}{fg=UFGblue} % TOC sections
% \setbeamercolor{title}{bg=UFGgrey}
% \setbeamercolor{item}{fg=green}
% \setbeamercolor{block body alerted}{bg=alerted text.fg!10}
% \setbeamercolor{block body}{bg=UFGblue!20}
% \setbeamercolor{block body example}{bg=UFGgreen!20}
%\setbeamercolor{block body alerted}{bg=UFGblue!10}
% \setbeamercolor{block title alerted}{bg=UFGred, fg=white}


% Transparencia dos blocks
%\addtobeamertemplate{block begin}{\pgfsetfillopacity{1}}{\pgfsetfillopacity{1}}
%\addtobeamertemplate{block alerted begin}{\pgfsetfillopacity{1}}{\pgfsetfillopacity{1}}
%\addtobeamertemplate{block example begin}{\pgfsetfillopacity{1}}{\pgfsetfillopacity{1}}


% % Override palette coloring with secondary
% \setbeamercolor{subsection in head/foot}{bg=UFGblue,fg=white}
% \setbeamercolor{section in head/foot}{bg=UFGgreen,fg=white}
% \setbeamertemplate{enumerate items}[default]
% \setbeamercolor{enumerate item}{fg=UFGblue}


% Desativando o cabeçalho
%\setbeamertemplate{headline}{}

% Colocando numero de paginas no slide
\setbeamertemplate{footline}{}

% Tela cheia
\hypersetup{pdfpagemode=FullScreen}

% justificando o texto nos blocks
\addtobeamertemplate{block begin}{}{\justifying}  %new code
\addtobeamertemplate{block example begin}{}{\justifying}  %new code
\addtobeamertemplate{block alerted begin}{}{\justifying}  %new code

% justificando o texto fora dos blocks
\apptocmd{\frame}{}{\justifying}{}

% Configurações da faixa nos frametitles
\makeatletter
\setlength{\metropolis@frametitle@padding}{1.3ex}%  <- largura da faixa
\setbeamertemplate{frametitle}{%
  \nointerlineskip%
  \begin{beamercolorbox}[%
      wd=\paperwidth,%
      sep=3pt,% 
      leftskip=\metropolis@frametitle@padding,%
      rightskip=\metropolis@frametitle@padding,%
    ]{frametitle}%
    \metropolis@frametitlestrut@start%
    \insertframetitle%
    \nolinebreak%
    \metropolis@frametitlestrut@end%
  \end{beamercolorbox}
}
\makeatother

\setbeamertemplate{itemize body end}{\vspace{2cm}}
\setbeamertemplate{itemize items}[default]

% Configuração dos frames
% \setbeamersize{
%     text margin left    = .06\paperwidth,
%     text margin right   = .03\paperwidth,
%     sidebar width left  = 0mm,
%     sidebar width right = 10mm,
%     description width   = 10mm,
%     mini frame size     = 10mm,
%     mini frame offset   = 10mm
% }

\makeatletter
\setlength\beamer@paperwidth{16.00cm}
\setlength\beamer@paperheight{10.00cm}
\geometry{%
  papersize={\beamer@paperwidth,\beamer@paperheight},
  hmargin=1cm,% 
  vmargin=0cm,%
  head=0.5cm,% might be changed later 
  headsep=0pt,%
  foot=1.5cm% might be changed later 
}
\makeatother

\counterwithout{figure}{section}
\counterwithout{figure}{subsection}
\counterwithout{table}{section}
\counterwithout{table}{subsection}



