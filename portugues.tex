
% Habilita a língua portuguesa
%\usepackage[utf8]{inputenc}
\usepackage[T1]{fontenc}
\usepackage[brazilian]{babel}


\newcommand{\sen}{\mathrm{sen \, }}
\newcommand{\senh}{\mathrm{senh \, }}
\newcommand{\tg}{\mathrm{tg\, }}
\newcommand{\cotg}{\mathrm{cotg\, }}
\newcommand{\arctg}{\mathrm{arctg \, }}
\newcommand{\arcsen}{\mathrm{arcsen \, }}
\newcommand{\Arg}{\mathrm{Arg \, }}
\newcommand{\cossec}{\mathrm{cossec\, }}
\newcommand{\cqd}{\hfill\square}
\newcommand{\dom}{\operatorname{Dom}}
\newcommand{\supt}{\operatorname{supt}}
\newcommand{\R}{\mathbb{R}}
\newcommand{\C}{\mathbb{C}}
\newcommand{\N}{\mathbb{N}}
\newcommand{\Z}{\mathbb{Z}}
\newcommand{\Q}{\mathbb{Q}}

%%%%%%%%%%%%%%%%%%%%%%%%%%%%%%%%%%%%%%%%%%%%
\theoremstyle{plain}
\newtheorem{teorema}{\blue{ Teorema\inserttheoremnumber}}
\newtheorem{corolario}{Corolário\inserttheoremnumber}
\newtheorem{lema}{Lema\inserttheoremnumber}
\newtheorem{axioma}{Axioma\inserttheoremnumber}
\newtheorem{propriedade}{Propriedade}
\newtheorem{proposicao}{Proposição\inserttheoremnumber}
\newtheorem{observacao}{Observação\inserttheoremnumber}

\newtheorem{question}{{\color{UFGblue}\rule{1cm}{2pt}}\ Questão}

%\newtheorem{question}{Questão}
\newtheorem{problema}{Problema\inserttheoremnumber}
\theoremstyle{definition} 
\newtheorem{definicao}{\blue{ Definição\inserttheoremnumber}}
\newtheorem{obs}{\red{ Observação\inserttheoremnumber}} 
\newtheorem{exemplo}{\green{ Exemplo\inserttheoremnumber}}
\newtheorem{exercicio}{\red{ Exercício\inserttheoremnumber}}

%%%%%%%%%%%%%%%%%%%%%%%%%%%%%%%%%%%%%%%%%%%%
\newenvironment{prova}{\vspace{.1cm}\noindent {\red{\bf Prova.}}}{
$\cqd$\vspace{2mm}}

\newenvironment{solucao}{\vspace{.1cm}\noindent {\green{\bf Solução.}}\footnotesize}{
$\cqd$\vspace{2mm}}

\columnsep=10pt % This is the amount of white space between the columns in the poster
\columnseprule=1pt % This is the thickness of the black line between the columns in the poster

%%%%%%%%%%%%%%%%%%%%%%%%%%%%%%%%%%%%%%%%%%%%


\usepackage{siunitx}

\sisetup{
  group-four-digits = true,
  group-separator = {.}
}

%transforma o simbolo decimal . em , no modo matemático.
% \DeclareMathSymbol{.}{\mathord}{letters}{"3B}