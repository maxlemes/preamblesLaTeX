 \usepackage{amsmath,amsthm,amsfonts,amssymb,amscd, amsxtra, mathrsfs}
 \usepackage{url}
 \usepackage{color}
 \usepackage{xcolor}
 \usepackage{lmodern}
 \usepackage{graphicx,float}
% \usepackage{pdfsync}
% \usepackage{lineno}
 \usepackage[labelfont=bf]{caption}
 \usepackage{siunitx} %% Sistema Internacional de Unidades
 \usepackage{multicol}
 \usepackage{subfigmat}
% \usepackage{changepage}
 \usepackage{hyperref}
% \usepackage{ragged2e}
 \usepackage[utf8]{inputenc} % codificacao de caracteres
 \usepackage[T1]{fontenc}    % codificacao de fontes
% \usepackage{lipsum}
 \usepackage{booktabs}
 \usepackage{colortbl}
% \usepackage{xspace}
 \usepackage{geometry}
% \usepackage[scale=2]{ccicons}
 \usepackage[portuguese]{babel} % Habilita a língua portuguesa
 \usepackage{answers}
 \usepackage{pst-poker} % cartas do baralho
\usepackage{enumerate}

\usepackage{tabularx}

\newcommand{\sen}{\mathrm{sen \, }}
\newcommand{\senh}{\mathrm{senh \, }}
\newcommand{\tg}{\mathrm{tg\, }}
\newcommand{\cotg}{\mathrm{cotg\, }}
\newcommand{\arctg}{\mathrm{arctg \, }}
\newcommand{\arcsen}{\mathrm{arcsen \, }}
\newcommand{\Arg}{\mathrm{Arg \, }}
\newcommand{\cossec}{\mathrm{cossec\, }}
\newcommand{\cqd}{\hfill\square}
\newcommand{\dom}{\operatorname{Dom}}
\newcommand{\supt}{\operatorname{supt}}
\newcommand{\R}{\mathbb{R}}
\newcommand{\C}{\mathbb{C}}
\newcommand{\N}{\mathbb{N}}
\newcommand{\Z}{\mathbb{Z}}
\newcommand{\Q}{\mathbb{Q}}

% definindo as margens do documento
\geometry{a4paper,text={17.5cm,25.2cm},centering}

%transforma o simbolo decimal . em , 
% \DeclareMathSymbol{.}{\mathord}{letters}{"3B}

\sisetup{locale = FR} % No Portuguese locale built-in, but French I think covers it

%Estilo de Página
%\pagestyle{headandfoot}
\pagestyle{empty}\runningheadrule
\firstpageheader{Left}{Center}{Right}
\runningheader{LeftM}{Middle, Page \thepage\ of \numpages}{RightM}
\firstpagefooter{}{}{}

%Substitui a palavra points por pontos.
\pointpoints{ponto}{pontos} %\pointpoints{SingularText}{PluralText}
%\bonuspointpoints{exam-bonus-point}{pontos}
% \qformat{\textbf{Questão \thequestion}\quad (\totalpoints\;\points)\hfill}

%Solução
\renewcommand{\solutiontitle}{\noindent\textbf{Solução:}\enspace}

%Verdadeiro ou Falso
%Customizável
\newcommand{\vf}[1][{}]{( \fillin[#1][0in] )}

%nomes no quadro de notas vertical		
\vqword{Questão}
\vpgword{Página}
\vpword{Valor}
\vsword{Nota}
\vtword{\textbf{Total}}

%nomes no quadro de notas horizontal
\hqword{Questão}
\hpgword{Página}
\hpword{Valor}
\hsword{Nota}
\htword{\textbf{Total}}

\definecolor{UFGred}{HTML}{990000}

\hypersetup{
    colorlinks=true,
    linkcolor=UFGred,
    filecolor=magenta,      
    urlcolor=cyan,
    pdftitle={Overleaf Example},
    pdfpagemode=FullScreen,
    }
