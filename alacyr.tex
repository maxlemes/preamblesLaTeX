%\usepackage[default]{comfortaa}
\usepackage{tcolorbox}
\usepackage{svg} 
\usepackage{cancel}
\usepackage{animate}
%\usepackage{pdfpages}


%%%%%%%%%%%%%%%%%%%%%%%%%%%%%%%%%%%%%%%%%%%%
\theoremstyle{plain}
\newtheorem{teorema}{\blue{ Teorema\inserttheoremnumber}}
\newtheorem{corolario}{Corolário\inserttheoremnumber}
\newtheorem{lema}{Lema\inserttheoremnumber}
\newtheorem{propriedade}{Propriedade}
\newtheorem{proposicao}{Proposição\inserttheoremnumber}
%\newtheorem{question}{Questão}
\newtheorem{problema}{Problema\inserttheoremnumber}
\theoremstyle{definition} 
\newtheorem{definicao}{\blue{ Definição\inserttheoremnumber}}
\newtheorem{obs}{\red{ Observação\inserttheoremnumber}} 
\newtheorem{exemplo}{\green{ Exemplo\inserttheoremnumber}}

\newtheorem{exercicio}{\red{ Exercício\inserttheoremnumber}}


\newenvironment{prova}{\vspace{.1cm}\noindent {\red{\bf Prova.}}}{
$\cqd$\vspace{2mm}}

\newenvironment{solucao}{\vspace{.1cm}\noindent {\green{\bf Solução.}}\footnotesize}{
$\cqd$\vspace{2mm}}

\newcommand{\cqd}{\hfill\square}

\columnsep=10pt % This is the amount of white space between the columns in the poster
\columnseprule=1pt % This is the thickness of the black line between the columns in the poster

%%%%%%%%%%%%%%%%%%%%%%%%%%%%%%%%%%%%%%%%%%%%


%%%%%%%%%%%%%%%%%%%%%%%%%%%% Definições das minhas cores %%%%%%%%%%%%%%%%%%%%%%%%%%%%%
\definecolor{mybcolor}{rgb}{0.122, 0.435, 0.698}
\definecolor{mygcolor}{rgb}{0.0, 0.7, 0.2}
\definecolor{myrcolor}{rgb}{0.8, 0.0, 0.2}

\definecolor{darkgreen}{rgb}{0.0509, 0.4509, 0.1568}
\definecolor{orangeb}{rgb}{0.9216, 0.4863, 0.0784}%{HTML}{EB811B}
\definecolor{blued}{rgb}{0.0039, 0.3529, 0.6431} 



\newcommand\bigzero{\makebox(0,0){\text{\huge0}}}
\newcommand*{\bord}{\multicolumn{1}{c|}{}}
\newcommand*{\bordd}{\multicolumn{1}{|c}{}}


\newcommand{\sen}{\mathrm{sen \!\, }}
\newcommand{\senh}{\mathrm{senh \, }}
\newcommand{\tg}{\mathrm{tg\, }}
\newcommand{\cotg}{\mathrm{cotg\, }}
\newcommand{\arctg}{\mathrm{arctg \, }}
\newcommand{\arcsen}{\mathrm{arcsen \, }}
\newcommand{\Arg}{\mathrm{Arg \, }}
\newcommand{\cossec}{\mathrm{cossec\, }}

\usepackage{pgf,tikz,pgfplots}
\usepackage{nicematrix}
\usetikzlibrary{matrix,arrows, decorations.pathreplacing, calc, positioning,fit,math}
%\pgfplotsset{compat=1.15}
\usepackage{mathrsfs}
%\pagestyle{empty}
%\usetikzlibrary{positioning}


% \newcommand*{\mybbox}{\box}
% \newcommand*{\mygbox}{\box}
% \newcommand*{\myrbox}{\box}


%%%%%%%%%%%%%%%%%%%%%%% Definições dos box %%%%%%%%%%%%%%%%%%%%%%%%%%%%%%%%%
\newtcbox{\mybbox}{nobeforeafter,colframe=mybcolor,colback=mybcolor!20!white,boxrule=1pt,arc=4pt,boxsep=0pt,left=0pt,right=4pt,top=0pt,bottom=16pt,tcbox raise base}
\usepackage[symbols,nogroupskip,nonumberlist]{glossaries-extra}

\newtcbox{\mmybbox}{nobeforeafter,colframe=mybcolor,colback=mybcolor!10!white,boxrule=0.5pt,arc=4pt,boxsep=0pt,left=6pt,right=6pt,top=1pt,bottom=1pt,tcbox raise base}
\usepackage[symbols,nogroupskip,nonumberlist]{glossaries-extra}


\newtcbox{\mygbox}{nobeforeafter,colframe=mybcolor,colback=orange!10!white,boxrule=1.0pt,arc=4pt,boxsep=0pt,left=6pt,right=6pt,top=4pt,bottom=4pt,tcbox raise base}
\usepackage[symbols,nogroupskip,nonumberlist]{glossaries-extra}


\newtcbox{\myrbox}{nobeforeafter,colframe=myrcolor,colback=myrcolor!10!white,boxrule=0.5pt,arc=4pt,boxsep=0pt,left=6pt,right=6pt,top=1pt,bottom=1pt,tcbox raise base}
\usepackage[symbols,nogroupskip,nonumberlist]{glossaries-extra}